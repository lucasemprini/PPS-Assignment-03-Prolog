%% TeXworks instructions:
% !TeX root = ./assignment.tex
% !TEX encoding = UTF-8 Unicode
% !TEX program = arara
% !TEX TS-program = arara
% !TeX spellcheck = en-US

% arara: pdflatex
% arara: pdflatex: { synctex: yes }

%% Generate a report.xmpdata file with title and authors for PDF/A-compliant format %%
\begin{filecontents*}{\jobname.xmpdata}
    \Title{Assignment 03 Prolog Report}
    \Author{Niccolò Maltoni}
\end{filecontents*}

\documentclass[
    a4paper,            % specifica il formato A4 (default: letter)
    10pt,               % specifica la dimensione del carattere a 10
    oneside,            % serve per impaginare per stampa solo fronte
    english
]{article}

\usepackage[T1]{fontenc}        % serve per impostare la codifica di output del font
\usepackage{textcomp}           % serve per fornire supporto ai Text Companion fonts
\usepackage[utf8]{inputenc}     % serve per impostare la codifica di input del font
\usepackage[english]{babel}
\usepackage{lmodern}            % carica una variante Latin Modern prodotto dal GUST

\usepackage[%
    strict,             % rende tutti gli warning degli errori
    autostyle,          % imposta lo stile in base al linguaggio specificato in babel
    english=american    % imposta lo stile per l'inglese
]{csquotes}                     % serve a impostare lo stile delle virgolette

\usepackage[%
    depth=3,            % equivale a bookmarksdepth di hyperref
    open=false,         % equivale a bookmarksopen di hyperref
    numbered=true       % equivale a bookmarksnumbered di hyperref
]{bookmark}                     % Gestisce i segnalibri meglio di hyperref
\usepackage{hyperref}           % Gestisce tutte le cose ipertestuali del pdf
\hypersetup{%
    pdfpagemode={UseNone},
    hidelinks,          % nasconde i collegamenti (non vengono quadrettati)
    hypertexnames=false,
    linktoc=all,        % inserisce i link nell'indice
    unicode=true,       % only Latin characters in Acrobat’s bookmarks
    pdftoolbar=false,   % show Acrobat’s toolbar?
    pdfmenubar=false,   % show Acrobat’s menu?
    plainpages=false,
    breaklinks,
    pdfstartview={Fit},
    pdfauthor={Luca Semprini},
    pdfcreator={Luca Semprini},
    pdftitle={Assignment 03 Prolog Report},
    pdflang={it}
}
\usepackage[a-1b]{pdfx}
\usepackage[%
    english,            % definizione delle lingue da usare
    nameinlink          % inserisce i link nei riferimenti
]{cleveref}                     % permette di usare riferimenti migliori dei \ref e dei varioref

% MARGINI LARGHI
\textwidth 6.3 in % Width of text line.
    \textheight 9.2 in
    \oddsidemargin 0 in      %   Left margin on odd-numbered pages.
    \evensidemargin 0 in      %   Left margin on even-numbered pages.
    \topmargin 0.2 in
    \headheight 0 in       %   Width of marginal notes.
    \headsep 0 in
    \topskip 0 in
    
\title{\vspace{-70pt}Assignment 03 -- ``Prolog''}
\author{Luca Semprini, matr: 854447, email: {\href{mailto:luca.semprini10@studio.unibo.it}{\texttt{luca.semprini10@studio.unibo.it}}}
\date{21/01/2018}


\begin{document}

    \maketitle
    \vspace{-30pt}

    \subsection*{Description of the code you provide}

        \begin{itemize}
            \item 
                Full solution to lab 10 along with final exercises: \texttt{inv}, \texttt{double}, \texttt{times}, \texttt{proj}.
                \begin{itemize}
                    \item Check comments on the \texttt{.pl} files
                    \item Necessary time: $\sim$~3.5 hours
                \end{itemize}

            \item 
                Full solution to lab 11.
                \begin{itemize}
                    \item Check comments on the \texttt{.pl} files
                    \item Necessary time: $\sim$~6 hours
                \end{itemize}
        \end{itemize}

    \subsection*{Techniques used}

        \begin{itemize}
            \item Prolog search-space abilities;
            \item Prolog use of unification;
            \item Prolog use of variables (use of universal facts);
            \item Prolog use of substitutions;
            \item Prolog use of rules;
            \item Prolog use of fully relationality;
            \item Prolog use of tail recursion;
            \item Prolog use of predicate 0-ary \texttt{cut} (\texttt{!});
            \item Prolog use of advanced library predicates like \texttt{findall/3}, \texttt{member/2} and \texttt{append/3};
            \item Java-tuProlog integration and testing.
        \end{itemize}

    \subsection*{Self-evaluation}
        In making of this assignment, I learned a lot of concepts and techniques of Prolog. 
        I was not skilled on LP at all, but solving these exercises I improved my abilities. 
        As I started understanding Prolog, I also started realizing its viable applications in real-life problems,
        such as in recursive problems, wich are very common.
 
\end{document}
    
    
